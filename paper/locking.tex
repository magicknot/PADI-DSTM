\subsection{Locking}
Para a obtenção de locks a classe \textit{PadInt}, referente a um dado \textit{PadInt} identificado por \textit{UID}, disponibiliza um conjunto de métodos cujo funcionamento será explicado de seguida. Estes métodos são utilizados pela classe \textit{Server} para responder a pedidos de leitura ou escrita.

\subsubsection{GetReadLock(TID)}
\label{GetReadLock}
Quando este método é invocado começa-se por verificar se o lock de leitura pretendido pela transação, identificada por \textit{TID}, já lhe foi atribuído ou se já lhe foi atribuído um lock de escrita. Nesse caso, retorna-se true. Caso contrário, é chamado o método AcqurireLock, descrito na secção \ref{obterLock}, com o tipo de lock pretendido, neste caso leitura.

\subsubsection{GetWriteLock(TID)}
\label{GetWriteLock}
Primeiro verifica-se se o lock de escrita pretendido pela transação, identificada por \textit{TID}, já lhe foi atribuído. Nesse caso, retorna-se true. Caso contrário, é chamado o método AcqurireLock, descrito na secção \ref{obterLock}, com o tipo de lock pretendido, neste caso escrita.

\subsubsection{AcquireLock(TID, requiredLockType)}
\label{obterLock}
Este método é invocado para obter locks de leitura ou escrita, consoante o valor do argumento \textit{requiredLockType}. O argumento \textit{TID} identifica a transação que está a tentar obter o lock.

Se for possível obter o lock do tipo pedido, a variável \textit{lockType} da classe\textit{PadInt} é actualizada para o novo tipo de lock, \textit{requiredLockType}. De seguida, caso o tipo de lock pedido seja leitura, o \textit{TID} é adicionado à variável \textit{readers}. Caso contrário, isto é o tipo de lock pedido é de escrita, é verificado se se trata de uma promoção, isto é, a transação já tinha o lock de leitura e pretende obter o lock de escrita e nesse caso, o \textit{TID} é removido da variável \textit{readers}. Por fim, é atribuído à variável \textit{writer} o valor \textit{TID}.

Quando não é possível obter o lock o pedido é posto em espera no máximo durante um determinado intervalo de tempo, usando o método \textit{Wait} da classe \textit{Monitor} disponibilizado pela linguagem C\#. Assim que é efectuado um \textit{commit} ou \textit{abort}, descrito nas secções \ref{commit} e \ref{abort}, todos os pedidos são retirados da fila de espera e aos que for possível tentar adquirir o lock este ser-lhes-á atribuído. Os restantes voltarão a ficar em espera. Se ao fim do intervalo de tempo máximo estipulado um pedido ainda se encontra em espera então foi detectado um \textit{deadlock} e esse pedido recebe uma excepção a indicar que tem que abortar.

Um pedido fica em espera num de dois casos. O primeiro é quando o tipo de lock actualmente atribuído e o lock pedido são ambos do tipo escrita. O segundo ocorre quando o tipo de lock actualmente atribuído e o pedido são diferentes e a transação não está a tentar realizar uma promoção possível, isto é, se o tipo de lock atribuído é de leitura e a transação está a tentar obter um lock de escrita e não é possível realizar a promoção porque não é só esta a transação que possui o lock de leitura.

\subsubsection{FreeWriteLock(TID)}
\label{FreeWriteLock}
O servidor remove o lock de escrita da transacção identificada por \textit{TID}, associado ao \textit{PadInt}.  Neste passo basta apenas inutilizar a variável \textit{writer}, atribuíndo-lhe um valor que não identique numa transação.

\subsubsection{FreeReadLock(TID)}
\label{FreeReadLock}
O servidor remove o lock de leitura, da transacção identificada pelo \textit{TID}, associado ao \textit{PadInt} identificado por \textit{UID}. Neste passo basta apenas remover o \textit{TID} da variável \textit{readers}.