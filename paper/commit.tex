\subsubsection{Commit}
\label{commit}

Quando é invocado o método \textit{TxCommit} da \textit{Library}, caso tenham sido criados ou acedidos \textit{PadInt}s, são efectuados dois passos. O primeiro passo corresponde à escrita de todos os valores escritos em \textit{Cache} e encontra-se descrito na secção \ref{cache}. No segundo passo, é enviado a cada \textit{Server} um pedido de commit contendo todos os \textit{UID} dos \textit{PadInt}s acedidos nesse \textit{Server} e a \textit{Cache} é re-iniciada.

O \textit{Server} ao receber este pedido verifica se guarda todos os \textit{PadInt}s identificados pelos \textit{UID} recebidos. Caso não guarde algum deles lança uma excepção a indicar que não guarda o \textit{PadInt}. Caso contrário, usando os métodos da classe \textit{PadInt} descritos nas secções \ref{FreeReadLock} e \ref{FreeWriteLock} são libertados os locks de leitura ou escrita atribuídos à transação que está a efectuar o commit.