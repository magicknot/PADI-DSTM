\subsection{Aborts}
Para identificar a taxa de aborts de transações submetemos a nossa solução a três testes. Antes de mais salientamos que a forma como estes testes foram realizados não tiram partido da utilização da \textit{Cache}, pois nunca são feitas duas ou mais leituras ou escritas no mesmo \textit{PadInt} durante uma transação. No primeiro destes testes os clientes realizaram 3 leituras e 3 escritas por iteração em três \textit{PadInts}, durante 300 iterações (por iteração entenda-se iniciar a transação, realizar as leituras e escritas e realizar o commit no final). Durante este teste variámos o número de clientes e pares de \textit{Server}s (primário, secundário) de forma a cobrir vários casos. Os resultados são apresentados na tabela \ref{figAAborts}. Como se pode verificar o número de aborts é bastante menor no caso em que existem apenas dois clientes e aumenta conforme o crescimento do número de clientes. Salientamos que neste teste todos os clientes efectuam leituras e escritas nos mesmos \textit{PadInt}s, sendo por isso normal o número de aborts observados.

\begin{table}[h]
\centering
\begin{tabular}{l|c|c|c|c|c|}
\cline{2-6}
                                 & \multicolumn{5}{c|}{N. de Clientes}                                                                                                   \\ \cline{2-6} 
                                 & \multicolumn{2}{c|}{2}                            & \multicolumn{3}{c|}{3}                                                      \\ \hline
\multicolumn{1}{|l|}{N. de Pares de Servers} & \multicolumn{1}{l|}{C1} & \multicolumn{1}{l|}{C2} & \multicolumn{1}{l|}{C1} & \multicolumn{1}{l|}{C2} & \multicolumn{1}{l|}{C3} \\ \hline
\multicolumn{1}{|c|}{1}          & 6                       & 6                       & 136                     & 125                     & 131                     \\ \hline
\multicolumn{1}{|c|}{2}          & -                       & -                       & 153                     & 154                     & 164                     \\ \hline
\end{tabular}
\caption{Número de aborts tendo 3 leituras e 3 escritas por iteração} \label{figAAborts}
\end{table}

Durante o segundo teste foram realizadas duas leituras e oito escritas por cada iteração, no total de dez. No terceiro teste foram realizadas oito leituras e duas escritas também no total de dez iterações. Os resultados apresentados nas tabelas \ref{figBAborts} e \ref{figCAborts} confirmam as conclusões suportadas pelo primeiro teste.

\begin{table}[h]
\begin{tabular}{c|c|c|c|c|c|}
\cline{2-6}
                                 & \multicolumn{5}{c|}{N. de Clientes}                   \\ \cline{2-6} 
                                 & \multicolumn{2}{c|}{2} & \multicolumn{3}{c|}{3} \\ \hline
\multicolumn{1}{|c|}{N. de Pares de Servers} & C1         & C2        & C1     & C2    & C3    \\ \hline
\multicolumn{1}{|c|}{1}          & 0          & 0         & 4      & 5     &       \\ \hline
\end{tabular}
\caption{Número de aborts tendo 2 leituras e 8 escritas por iteração} \label{figBAborts}
\end{table}

\begin{table}[h]
\begin{tabular}{c|c|c|c|c|c|}
\cline{2-6}
                                 & \multicolumn{5}{c|}{N. de Clientes}                   \\ \cline{2-6} 
                                 & \multicolumn{2}{c|}{2} & \multicolumn{3}{c|}{3} \\ \hline
\multicolumn{1}{|c|}{N. de Pares de Servers} & C1         & C2        & C1     & C2    & C3    \\ \hline
\multicolumn{1}{|c|}{1}          & 1          & 0         & 4      & 5     & 2     \\ \hline
\end{tabular}
\caption{Número de aborts tendo 8 leituras e 2 escritas por iteração} \label{figCAborts}
\end{table}