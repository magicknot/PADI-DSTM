\subsection{Commit e abort}

Antes de explicarmos como funciona o commit e o abort explicaremos uma opção tomada em relação ao uso de two-phase-commit (2PC). Tendo em conta o âmbito deste projecto e a abordagem pessimista que decidimos seguir, tal como descrevemos na secção \ref{Optimista vs Pessimista}, não precisamos de usar 2PC.

Esta conclusão é baseada no facto que quando um cliente deseja realizar commit ou abort este já obteve previamente todos os locks que necessitou e ainda não os libertou. Devido a isto quando um cliente deseja realizar commit ou abort, mesmo que tenha efectuado vários pedidos a \textit{Server}s diferentes basta que faça commit ou abort em cada um desses \textit{Server}s de forma a libertar os locks só passando ao \textit{Server} seguinte quando obtiver a confirmação que o commit ou abort num dado \textit{Server} foi bem sucedido.

Mesmo no caso em que um \textit{Server} falha ou os \textit{PadInt}s são redistribuidos para outro \textit{Server} em consequência de ter sido criado um novo par (primário, secundário) o cliente quando recebe uma excepção a informar que o que o servidor não respondeu ou que o \textit{PadInt} não foi encontrado, respectivamente, este pergunta novamente ao \textit{Master} qual é o \textit{Server} que guarda actualmente o \textit{PadInt}. Depois disto volta a re-enviar o pedido e irá obter uma resposta, pois ou o \textit{Server} foi recuperado devido à replicação passiva ou o cliente agora sim enviou o pedido para o \textit{Server} que guarda o \textit{PadInt}.

\subsection{Commit}
\label{commit}

Quando é invocado o método TxCommit da \textit{Library}, caso tenham sido criados ou acedidos \textit{PadInt}s, são efectuados dois passos. O primeiro passo corresponde à escrita de todos os valores escritos em \textit{Cache} e encontra-se descrito na secção \ref{cache}. No segundo passo, é enviado a cada \textit{Server} um pedido de commit contendo todos os \textit{UID} dos \textit{PadInt}s acedidos nesse \textit{Server} e a cache é re-iniciada.

O \textit{Server} ao receber este pedido verifica se guarda todos os \textit{PadInt}s identificados pelos \textit{UID} recebidos. Caso não guarde algum deles lança uma execepção a indicar que não guarda o \textit{PadInt}. Caso contrário, usando os métodos da classe \textit{PadInt} descritos nas secções \ref{FreeReadLock} e \ref{FreeWriteLock} são libertados os locks de leitura ou escrita atribuídos à transação que está a efectuar o commit.
\subsubsection{Abort}
\label{abort}

O método \textit{TxAbort} da \textit{Library} é em tudo semelhante ao método \textit{TxCommit} excepto em dois pontos. A primeira diferença é que antes ser libertádo cada lock de escrita associado a cada \textit{UID} referenciado pela transação, o valor actual do \textit{PadInt} é substituído pelo valor registado como sendo o valor original antes da transação o ter alterado, isto é, é reposto o valor do último commit realizado com sucesso. A segunda diferença está relacionada com a \textit{Cache}, pois não existe a escrita dos valores nos \textit{Server}s, devido à primeira diferença enunciada.
