\section{Solução}

\subsection{Optimista vs Pessimista}
\label{Optimista vs Pessimista}
Inicialmente considerámos seguir uma abordagem optimista. No entanto, não conhecendo a relação existente entre o número de leituras e escritas e admitindo que existe um igual número de leituras e de escritas, o número de conflitos poderá ser grande. 

Constatando esse potencial bottleneck de performance no nosso sistema e o facto das transacções poderem abortar (se surgir algum conflito) depois de já terem realizado trabalho, o que levaria a refazer esse trabalho, optámos por abandonar as soluções optimistas.

Assim, escolhemos utilizar \textit{Strict Two Phase Locking} ou \textit{S2PL}.

\subsection{Replicação activa vs replicação passiva}

Depois de escolhermos qual o protocolo a usar, deparámo-nos com a escolha entre replicação activa e passiva. Inicialmente considerámos usar replicação activa com um protocolo de \textit{Quorum Consensus}. No entanto, por esta precisar de mais servidores do que os necessários para a replicação passiva, e tendo também em conta que teríamos que enviar e receber mais mensagens do que as necessárias ao usar replicação passiva, optámos por esta última opção.
