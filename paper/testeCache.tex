\subsection{Cache}
Para demonstrar o efeito positivo que a \textit{Cache} tem na nossa solução apresentamos de seguida duas tabelas que mostram as diferenças de usar ou não a \textit{Cache}. Este exemplo pressupõe a existência de um cliente que acede a dois \textit{PadInts} guardados no mesmo \textit{Server} e efectua uma sequência de 300 leituras e escritas a intercaladas a cada um dos \textit{PadInts}.

Os dados apresentados nas tabelas confirmam que o número de pedidos que o \textit{Server} recebe com o uso da \textit{Cache} é muito menor do que quando não é utilizada uma \textit{Cache}. Para além destes dados foi também possível verificar que ao usar a \textit{Cache}  o tempo consumido pelo cliente foi menor relativamente ao caso em que a \textit{Cache} não foi usada.

É importante referir que estes efeitos se verificam sempre para leituras. Em relação às escritas estes resultados apenas se verificam para um número mínimo de duas escritas. Quando o cliente faz apenas uma escrita devido ao uso da \textit{Cache} serão efectuadas duas escritas, uma para obter o lock de escrita e outra antes de efectuar o commit.

\begin{table}[H]
\centering
\begin{tabular}{| p{1.5cm} | p{1.5cm} | p{1.5cm} | p{1.5cm} |}
\hline
\textbf{UID do \textit{PadInt}} & \textbf{Leituras no \textit{Server}} & \textbf{Escritas no \textit{Server}} & \textbf{Total} \\
\hline
1 & 300 & 300 & 600  \\
\hline
2 & 300 & 300 & 600  \\
\hline
Total & 600 & 600 & 1200  \\
\hline
\end{tabular}
\caption{Número de pedidos ao Server sem uso de Cache} \label{figAcache}
\end{table}

\begin{table}[H]
\centering
\begin{tabular}{| p{1.5cm} | p{1.5cm} | p{1.5cm} | p{1.5cm} |}
\hline
\textbf{UID do \textit{PadInt}} & \textbf{Leituras no \textit{Server}} & \textbf{Escritas no \textit{Server}} & \textbf{Total} \\
\hline
1 & 1 & 2 & 3  \\
\hline
2 & 1 & 2 & 3  \\
\hline
Total & 2 & 4 & 6  \\
\hline
\end{tabular}
\caption{Número de pedidos ao Server com uso de Cache} \label{figBcache}
\end{table}